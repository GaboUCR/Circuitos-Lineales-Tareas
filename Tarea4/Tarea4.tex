\documentclass{article}
\usepackage[utf8]{inputenc}
\usepackage{amsmath}
\usepackage{graphicx}
\graphicspath{ {./images/} }

\title{Tarea 04 de Circuitos Lineales I}
\author{Gabriel Gamboa Vargas}
\date{Noviembre 2021}

\begin{document}
\maketitle
\section{Ejercicio 1 y 4}
Se utiliza el programa Micro Cap. Se juntan ambos ejercicios para tener más claridad.\\
Se referirá a las corrientes de rama por el nombre del elemento de circuito por el cual la corriente circula. Se muestra el circuito por partes.\\
 Los valores en forma tabulada son:\\ \\    
\includegraphics[]{images/tensionesMicroCap.PNG}\\ 
A continuación se muestra donde están ubicados estos valores y se utilizan para calcular las corrientes de rama.\\
\includegraphics[]{images/MicroCap1_75.PNG}\\ 
El nombre del nodo está escrito en la izquierda y su tensión de nodo luego de los dos puntos.\\ \\
$I(F1) =  70.62m A$\\
$I(R1) = v1/R1 = 845u A$\\
$I(R2) = v2/R3 = -247.074 A$\\
$I(R3) = v3/R1 = 8.291m A$\\
$I(R4) = \frac{v4-v1}{R2} = -281.748u A$\\
$I(R5) = \frac{v3-v4}{R1} = 5.191m A$\\
$I(R7) = \frac{v4-v2}{R3} = 10.169m A$\\
$I(R8) = \frac{v4-v2}{R1} = 563.496u A$\\
$I(R9) = \frac{v5-v3}{R2} = 6.741m A$\\
$I(R10) = \frac{v5-v6}{R3} = 22.323m A$\\
$I(R11) = \frac{v4-v6}{R1} = 48.297m A$\\
$I(R12) = \frac{v4-v9}{R2} = 16.5m A$\\
$I(R13) = \frac{v2-v9}{R1} = 2.494m A$\\
$I(R14) = \frac{v2-v9}{R2} = 1.247m A$\\
$I(R15) = \frac{v5-v3}{R2} = 6.741m A$\\
$I(R16) = \frac{v5-v9}{R1} = 51.674m A$\\ \\
\includegraphics[]{images/MicroCap2_75.PNG}\\ \\
$I(G1) = 169.836m A$\\
$I(H1) = 20.917m A$\\
$I(R17) = \frac{v5-v19}{R1} = 52.219m A$\\ 
$I(R18) = \frac{v5-v11}{R3} = 139.699m A$\\ 
$I(R19) = \frac{v12-v11}{R1} = 104.838m A$\\ 
$I(R48) = \frac{v12-v11}{R3} = 34.946m A$\\ 
$I(R22) = \frac{v10-v15}{R2} = 26.118m A$\\ 
$I(R23) = \frac{v10-v17}{R2} = 26.101m A$\\ 
$I(R21) = \frac{v16-v13}{R1} = 122.801m A$\\ 
$I(R20) = \frac{v15-v13}{R3} = 47.035m A$\\ 
$I(R24) = \frac{v17-v19}{R3} = 51.674m A$\\ 
\includegraphics[]{images/MicroCap3_75.PNG}\\ \\
$I(E1) = 72.748m A$\\
$I(v1) = 75.142m A$\\
$I(R25) = \frac{v19-v16}{R1} = 2.789m A$\\ 
$I(R26) = \frac{v19-v20}{R2} = 2.395m A$\\ 
$I(R27) = \frac{v21-v20}{R1} = 72.748m A$\\ 
$I(R28) = \frac{v2-v9}{R3} = 831.45u A$\\ 
$I(R29) = \frac{v2-v16}{R2} = 36.791m A$\\ 
$I(R30) = \frac{v8-v16}{R1} = 117.7m A$\\ 
$I(R31) = \frac{v8-v2}{R2} = 22.059m A$\\ 
$I(R32) = \frac{v8-v16}{R3} = 39.233m A$\\ 
\includegraphics[]{images/MicroCap4_75.PNG}\\ \\
$I(R33) = \frac{v18-v16}{R2} = 58.85m A$\\ 
$I(R34) = \frac{v22-v8}{R1} = 237.842m A$\\ 
$I(R35) = \frac{v23-v22}{R3} = 76.35m A$\\ 
$I(R36) = \frac{v23-v22}{R2} = 114.525m A$\\ 
$I(R37) = \frac{v24-v22}{R3} = 46.968m A$\\ 
$I(R38) = \frac{v23-v16}{R2} = 39.233m A$\\ 
$I(R39) = \frac{v25-v24}{R2} = 35.658m A$\\ 
$I(R40) = \frac{v26-v24}{R3} = 11.31m A$\\ 
$I(R41) = \frac{v23-v25}{R1} = 16.83m A$\\ 
$I(R42) = \frac{v27-v25}{R1} = 18.828m A$\\ 
$I(R43) = \frac{v27-v28}{R1} = 11.225m A$\\ 
$I(R44) = \frac{v12-v27}{R2} = 30.053m A$\\ 
$I(R45) = \frac{v14-v26}{R1} = 11.31m A$\\ 
$I(R46) = \frac{v28-v14}{R3} = 11.225m A$\\ 
\includegraphics[]{images/MicroCap5_67.PNG}\\ \\
$I(R47) = \frac{v11-v14}{R2} = 84.918u A$\\ 
\section{Ejercicio 2}
Se utiliza el programa Ltspice, note que las tensiones de nodos están nombradas de la misma forma que en el circuito diseñado en Micro Cap, los resistores no tienen la misma numeración.\\ \\
Los valores de las tensiones de nodo son:\\ 
\includegraphics[]{images/ltspicetensiones.PNG}\\ \\
Puede encontrar donde se encuentran estos valores en las siguientes capturas.\\
\includegraphics[]{images/ltspice1_75.PNG}\\ \\
\includegraphics[]{images/ltspice2_75.PNG}\\ \\
\includegraphics[]{images/ltspice3_70.PNG}\\ \\
\section{Ejercicio 3}
Se utiliza el siguiente archivo SPICE.\\
Ejercicio 3\\
E1 21 9 14 11 2\\
F1 4 6 VF1 1.2\\
G1 12 13 14 11 0.1\\
H1 15 17 VH1 {6}\\
I1 16 23 DC 0.5\\
R1 0 1 30000\\
R2 2 0 30000\\
R3 3 0 10000\\
R4 4 1 20000\\
R5 3 4 10000\\
R7 4 2 30000\\
R8 1 4 10000\\
R9 3 5 20000\\
R10 6 5 30000\\
R11 4 6 10000\\
R12 4 9 20000\\
R13 2 9 10000\\
R14 2 9 20000\\
R15 3 5 20000\\
R16 9 5 10000\\
R17 10 5 10000\\
R18 5 11 30000\\
R19 12 11 10000\\
R20 15 13 30000\\
R21 16 13 10000\\
R22 15 10 20000\\
R23 17 10 20000\\
R24 19 17 30000\\
R25 19 16 10000\\
R26 20 19 20000\\
R27 21 20 10000\\
R28 2 9 30000\\
R29 2 16 20000\\
R30 8 16 10000\\
R31 2 8 20000\\
R32 8 16 30000\\
R33 18 16 20000\\
R34 8 22 10000\\
R35 22 23 30000\\
R36 22 23 20000\\
R37 22 24 30000\\
R38 16 23 20000\\
R39 25 24 20000\\
R40 24 26 30000\\
R41 23 25 10000\\
R42 25 27 10000\\
R43 27 28 10000\\
R44 27 12 20000\\
R45 26 14 10000\\
R46 28 14 30000\\
R47 14 11 20000\\
R48 12 11 30000\\
RE1 14 11 1G;added by E1\\
RG1 14 11 1G ;added by G1\\
V1 16 20 DC 20\\
VF1 7 8 0 ;added by F1\\
VH1 18 7 0 ;added by H1\\
*\\

*\\
*\\
.DC 0 0 0 0\\
*\\
.PROBE\\
.END\\

Se utiliza el comando "source two.cir" para cargar el circuito, una vez cargado exitosamente se usa el comando "op" para simular corriente continua. Después se usa el comando "print all" para mostrar todas las tensiones de nodo, note que está enumerado de la misma forma que el circuito de Ltspice y el de Micro Cap.\\
\includegraphics[]{images/ngspice1_70.PNG}\\ \\
$e+x$ representa un factor de $10^x$ \\  
\includegraphics[]{images/ngspice2.PNG}\\ \\
\end{document}